%%%%%%%%%%%%%%%%%%%%%%%%%%%%%%%%%%%%%%%%%%%%%%%%%%%%%%%%%%%%%%%%%%%%%%%%%%%%%%%%%%%%%%%%%%%
%%%% Die Dokumentenklasse und ihre Optionen %%%%%%%%%%%%%%%%%%%%%%%%%%%%%%%%%%%%%%%%%%%%%%%
\documentclass[captions=tableheading, 12pt, headings=small, parskip=half]{scrartcl}

%%%%%%%%%%%%%%%%%%%%%%%%%%%%%%%%%%%%%%%%%%%%%%%%%%%%%%%%%%%%%%%%%%%%%%%%%%%%%%%%%%%%%%%%%%%
%%%% LaTeX-Pakete %%%%%%%%%%%%%%%%%%%%%%%%%%%%%%%%%%%%%%%%%%%%%%%%%%%%%%%%%%%%%%%%%%%%%%%%%
\usepackage[utf8]{inputenc}
\usepackage[T1]{fontenc}
\usepackage[ngerman]{babel}
\usepackage{setspace,               % Zeilendurchschuss verändern (\doublespacing,\onehalfspacing)
            booktabs,               % Schöne Tabellen
            amsmath,                % Mathepackage der American Mathematical Society
            amsfonts,               % Paket für schönere mathematische Schriftarten
            amssymb,                % Paket für schönere mathematische Symbole
            bbm, 					% Blackboard-Style im Mathemodus
            bm, 					% Bold-Symbols im Mathemodus
            enumitem, 				% Anpassungen der enumerate-Umgebung
            graphicx,               % Paket zum Laden von Graphiken
            lmodern,                % Schrifttyp, der mit Microtype zusammenarbeitet
            nicefrac,				% Schräge Brüche
            csquotes,				% für Anführungszeichen
            microtype,              % Typographische Korrekturen bei Umbrüchen
            hyperref,				% url-Befehl
            calligra,
            setspace,
            multirow,
            float,
            eurosym,
            epstopdf,
            lscape,
            longtable,
            calc,
            dsfont
}
\usepackage[left=2.5cm,right=2.5cm,top=2cm,bottom=2cm]{geometry} % Seiteneinrichtung
%\usepackage{geometry} % Seiteneinrichtung

\usepackage{dcolumn}
\newcolumntype{d}[1]{D{.}{.}{#1}}


% \graphicspath{{C:\Users\stapperm\Documents\Econometrics 1\Code}}

\newcounter{tfno}
\newcommand{\mybox}{\resizebox{.5cm}{!}{\raisebox{-.5ex}{$\Box$}}}
%
\setcounter{tfno}{0} %% moved here
\newenvironment{truefalse}{%
	%\setcounter{tfno}{0}  %% moved up
	\renewcommand\arraystretch{1.5}
	\setlength\LTleft{0pt}
	\setlength\LTright{0pt}
	\begin{longtable}{>{\stepcounter{tfno}\thetfno.}cp{.5\textwidth}@{\extracolsep{\fill}}cc}
		\multicolumn{1}{r}{}&  & \fbox{\parbox{\widthof{True}}{True}} & \fbox{\parbox{\widthof{False}}{False}}  \\
	}{%
	\end{longtable}
	\renewcommand\arraystretch{1}
}
\newcommand\tfquestion[1]{ & #1 & \mybox  & \mybox  \\}


\begin{document}

\begin{table}[H]
	\begin{tabular}{lr}
		& \multirow{4}{*}{\includegraphics[width = 8cm]{Code2/wwu_logo.png}}\\
		Dr. Willi Mutschler& \\
		M.Sc. Manuel Stapper & \\
		Summer term 2018 \hphantom{MMMMMMMMMMM}& 
	\end{tabular}
\end{table}
\vspace{1cm}
\begin{center}
	{\Large Econometrics II} \\
	-- Comprehension Exercises --
\end{center}

\section*{\underline{Exercise 1}}

The food expenses $y_t$ of $n=150$ households are regressed on household income
$x_t$ and squared household income $x_t^2$. You should investigate if there
is heteroskedasticity. There are three groups: The poorest third of the households,
the middle third and the richest third. Regressions for the first and third
group yield the following sums of squared residuals:
\begin{align*}
S_{\hat u \hat u}^I &=     0.00838 \\
S_{\hat u \hat u}^{III} &= 0.07346.
\end{align*}

\begin{enumerate}[label = \alph*)]
	\item What might be a problem with classical OLS estimation in this situation?
	\item Perform a suitable test to check if the aforementioned problem is present.
	\item How would you proceed if the above test suggests the problem is present? 
\end{enumerate}

\section*{\underline{Exercise 2}}

Consider the model
\[ y=X\beta+u \qquad\textrm{with}\qquad u\sim N(0,\Omega) \]
with the GLS estimator
\[ \hat\beta_{GLS}=(X'\Omega^{-1}X)^{-1}X'\Omega^{-1}y. \]
\begin{enumerate}[label = \alph*)]
	\item Show that $\hat\beta_{GLS}$ is still unbiased if
	$X$ is stochastic but independent of $u$. You may assume that the
	matrix $\Omega$ is non-stochastic and known.
	\item Which conditions must be satisfied to ensure consistency of the 
	estimator?
\end{enumerate}

\section*{\underline{Exercise 3}}

Consider the simple linear regression model
\[ y_t=\alpha+\beta x_t+u_t. \]
\begin{enumerate}[label = \alph*)]
	\item Explain why an OLS estimation of the coefficients 
	yields inconsistent estimators if the regressor is correlated 
	with the error term.
	\item Is $x_{t-1}$ a valid instrument if a large (small)
	error in $t-1$ tends to be followed by a large (small)
	error in $t$?
\end{enumerate}

\section*{\underline{Exercise 4}}
\begin{enumerate}[label = \alph*)]
	\item In a regression model with errors that are non-normally distributed, what concept is the justification for tests to give valid results at large samples?
	\item Describe the principle of the \textit{Law of Large Numbers} in your own words.
	\item State the null hypothesis of the \textit{White-Test}. What is the advantage of it compared to the \textit{Goldfeld-Quandt-Test}?
	\item Is there a way to deal with possible heteroscedasticity without using the aforementioned tests?
	\item When not using a computer for the one-sided \textit{Durbin-Watson-Test}: Why is a lower and upper quantile for the distribution of test statistic needed?
	\item What values of a test statistics indicate positive/negative autocorrelation at a \textit{Durbin-Watson-Test}?
\end{enumerate}




\end{document}