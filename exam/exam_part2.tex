% !TeX encoding = UTF-8
% !TeX spellcheck = en_US
\documentclass{article}
%%%%%%%%%%%%%%%%%%%%%%%%%%%%%%%%%%%%%%%%%%%%%%%%%%%%%%%%%%%%%%%%%%%%%%%%%%%%%%%%%%%%%%%%%%%%%%%%%%%%%%%%%%%%%%%%%%%%%%%%%%%%%%%%%%%%%%%%%%%%%%%%%%%%%%%%%%%%%%%%%%%%%%%%%%%%%%%%%%%%%%%%%%%%%%%%%%%%%%%%%%%%%%%%%%%%%%%%%%%%%%%%%%%%%%%%%%%%%%%%%%%%%%%%%%%%
\usepackage[a4paper,top=2cm]{geometry}
\usepackage{amssymb,amsmath,amsfonts}
\usepackage[english]{babel}
\usepackage[a4paper]{geometry}
\usepackage{enumitem}
\usepackage{booktabs}
\usepackage{csquotes}
\usepackage{dcolumn}
\usepackage[T1]{fontenc}
\usepackage[utf8]{inputenc}
\newcolumntype{d}[1]{D{.}{.}{#1}}

\begin{document}
	
	\title{Econometrics 2 \\ \small Exam}
	\author{Dr. Willi Mutschler}
	\date{}
	\maketitle
	
	\begin{itemize}
		\item Answer \textbf{all} of the following exercises in either German or English.
		\item Explain your answers and derivations. All your computations and intermediate steps need to be verifiable and understandable. 
		\item Formulas which we covered in the lecture and class need not to be derived again.
		\item If you prefer a notation different from the one used in the course, define it.
		\item Always use significance level $a=5\%$.
		\item Please report 3 decimal places in numerical answers.
		\item If not otherwise stated, assume the validity of the assumption A, B and C given in the lecture.
		\item Permissible aids:
		\begin{itemize}
			\item non-programmable pocket calculator
			\item cheat sheet: one-sided A4 white sheet of paper with annotations, formulas, texts, sketches, etc.
		\end{itemize}
	\end{itemize}
\thispagestyle{empty}

\newpage
\setcounter{page}{1}
\section{Understanding}
\begin{enumerate}[label=(\alph*)]
\item The standard output of an OLS estimation yields a Durbin-Watson statistic of 0.32. What does this imply?
\item The estimation of a dynamic model yields
$$y_t = 0.6 y_{t-1} + 0.3 x_t +\hat{u}_t$$
where $y_t$ and $x_t$ are \textbf{both measured in logs}. Compute the dynamic effect (multiplier) of a 1\% increase in x on y (i) in the same period (ii) in the next period and (iii) in the long run.
\item List the assumptions on the error term one needs to show that the covariance matrix of the OLS estimator $\hat{\beta}$ is given by $\sigma^2 (X'X)^{-1}$?
\end{enumerate}
\newpage


\section{Violation of B1}
Assume that the true model is given by
$$y_t^\ast = \alpha + \beta x_t + u_t$$
where $u_t \sim N(0,\sigma_u^2)$ with $\sigma_u^2>0$. However, $y_t^\ast$ is only observable with a measurement error
$$y_t^\ast = y_t + v_t$$
where $v_t \sim N(0,\sigma_v^2)$ with $\sigma_v^2>0$ and $cov(u_t,v_s)=0$ for all $t$ and $s$. The observed model is hence given by
$$ y_t = \alpha + \beta x_t + \underbrace{u_t - v_t}_{\varepsilon_t}$$
\begin{enumerate}[label=(\alph*)]
	\item Compute the (i) expectation, (ii) variance and (iii) autocorrelation function of $\varepsilon_t$.
	\item Compute the bias of the OLS estimators for $\alpha$ and $\beta$.
	\item What happens to the standard errors of the OLS estimators for $\alpha$ and $\beta$? Are hypothesis tests and interval estimators still valid? 
	\item Provide intuition whether or not the OLS estimator remains consistent. A formal proof is not necessary.	
\end{enumerate}

\newpage
\section{Testing violations}
The demand for money is determined using the regression model
$$ m_t = \alpha + \beta_1 y_{t} + \beta_2 i_{t} + u_t$$
where
\begin{align*}
m_t:    & \text{ real money stock in logs}\\
y_{t}: & \text{ real income in logs}\\
i_{t}: & \text{ interest rate in \%}
\end{align*} 
Given quarterly data for the period 1970-1996 ($T=108$), a least squares estimation shows that
$$\sum_{t=1}^T \hat{u}_t^2=108$$
A separate estimation for the period 1970-1989 (80 observations) yields an unbiased estimate for $\sigma^2$ of 0.02, whereas considering the period 1990-1996 (28 observations) yields an unbiased estimate for $\sigma^2$ of 2. Test the following null hypotheses:
\begin{enumerate}[label=(\alph*)]
	\item The variance $\sigma^2$ did not change after the reunification 1990.
	\item The coefficients of the regression model did not change after the reunification 1990.
\end{enumerate}
\newpage

\section{Instrument variables}
The aim is to investigate the influence of education, $educ$, on a person's salary, $wage$, by using the following regression model
\[\log(wage_t) = \beta_0 + \beta_1\, educ_t + u_t\]
Since it is assumed that $educ$ is an endogenous variable, the geographical distance of the person's place of residence to the nearest university, $nearc4$, and the number of years of the father's education, $fatheduc$, are used as instrumental variables. The \textbf{IV estimation} yields the following result

\renewcommand{\arraystretch}{0.8}
\begin{center}
	
	Model: IV Estimation, observations 1--2320\\
	Endogenous Variable: log(wage)\\
	Instruments: const nearc4 fatheduc \\
	Instruments used for: educ \\	
	\vspace{1em}
	
	\begin{tabular}{lr@{,}lr@{,}lr@{,}lr@{,}l}
		&
		\multicolumn{2}{c}{Coefficient} &
		\multicolumn{2}{c}{Std.\ error} &
		\multicolumn{2}{c}{$t$ statistic} &
		\multicolumn{2}{c}{p-value} \\[1ex]
		const &
		5&3155 &
		0&0966 &
		55&0198 &
		0&0000 \\
		educ &
		0&0715 &
		0&0071 &
		10&0722 &
		0&0000 \\
	\end{tabular}
	
	\vspace{1em}
	
\end{center}

\renewcommand{\arraystretch}{1.5}

\begin{enumerate}[label=(\alph*)]
	\item What is an endogenous regressor? What is a relevant and valid instrument?
	
	\item An  OLS estimation of 
	$$ educ_t = \pi_0 + \pi_1 nearc4_t + \pi_2 fatheduc_t + v_t$$ yields
	\renewcommand{\arraystretch}{0.8}
	\begin{center}		
		Model: OLS, observations 1--2320\\
		Endogenous Variable: educ\\
		
		\vspace{1em}
		
		\begin{tabular}{lr@{,}lr@{,}lr@{,}lr@{,}l}
			&
			\multicolumn{2}{c}{Coefficient} &
			\multicolumn{2}{c}{Std.\ error} &
			\multicolumn{2}{c}{$t$-statistic} &
			\multicolumn{2}{c}{p-value} \\[1ex]
			const &
			10&0549 &
			0&1465 &
			68&6545 &
			0&0000 \\
			nearc4 &
			0&3388 &
			0&1038 &
			3&2653 &
			0&0011 \\
			fatheduc &
			0&3269 &
			0&0129 &
			25&3125 &
			0&0000 \\
		\end{tabular}
		
		\vspace{1ex}
		\begin{tabular}{lr}
		$F(2, 2317)$ = 344,2389, & p-value($F$) = 1,3\textrm{e--131}
		\end{tabular}
		
		\vspace{1em}
		
	\end{center}
	Are the instruments $nearc4$ and $fatheduc$ relevant? 
	\renewcommand{\arraystretch}{1.5}
	
	
\end{enumerate}

\newpage
\section{Stochastic convergence}
Suppose that the infinite sequence $Y_{1},Y_{2},\ldots $ is an $AR(1)$ process, i.e.
$$Y_{t}-\mu =\rho \left(Y_{t-1}-\mu\right) +\varepsilon _{t}$$ where $\varepsilon _{t}\sim iid(0,\sigma_{\varepsilon }^{2})$ and $|\rho |<1$. Furthermore, let $\hat{\mu} =\frac{1}{T}\sum_{t=1}^{T}Y_{t}$ denote the sample mean.\\
\textit{Hint: From the lecture we know that $Y_t$ is stationary, so that $E(Y_t) = \mu$ and $Var(Y_t) = \sigma_\varepsilon^2/(1-\rho^2)$ for all $t=1,...,T$.}

\begin{enumerate}[label=(\alph*)]
		\item Show that $\frac{1}{\sqrt{T} } \sum_{t=1}^T \varepsilon_t \overset{d}{\rightarrow} U_\varepsilon \sim N(0,\sigma_\varepsilon^2)$.
		\item Show that
		$$\frac{1}{\sqrt{T}} \sum_{t=1}^T \varepsilon_t = \sqrt{T}\left[(1-\rho)\left(\hat{\mu}-\mu\right) + \rho\left(\frac{Y_T - Y_0}{T}\right)\right]$$

		\textit{Hint: $\frac{1}{T}\sum_{t=1}^{T}Y_{t-1} = \left(\frac{1}{T}\sum_{t=1}^{T} Y_{t}\right) - \frac{1}{T} (Y_T - Y_0) = \hat{\mu} - \frac{Y_T - Y_0}{T}$.}
		\item Assume that
		$$\underset{T\rightarrow \infty}{\textsl{plim}}\left[\frac{\rho}{1-\rho}\left(\frac{Y_T - Y_0}{\sqrt{T}}\right)\right] = 0$$
		Put the results of (b) and (c) together and show that
		\begin{align}
		Z_{T} =\sqrt{T}\frac{\hat{\mu} -\mu }{\sigma_Z} \overset{d}{\rightarrow} U \sim N(0,1) \label{eq:AsymDistribAR1}
		\end{align}
		for $\sigma_Z=\sqrt{\sigma_\varepsilon^2/(1-\rho)^2}$.\\		
		\textit{Hint: Start with (c) and divide by $(1-\rho)$, then use (b) to derive the asymptotic distribution.}
		\item Briefly describe the intuition and result of the usual Central Limit Theorem for the sample mean of iid random variables. Why does it not hold for the $AR(1)$ process?\\
		\textit{Hint: Note that $\sigma_Z =\sqrt{\sigma_\varepsilon^2/(1-\rho)^2}$ \textbf{is not equal to} $\sqrt{Var(Y_t)} = \sqrt{\sigma_\varepsilon^2/(1-\rho^2)}$.}
\end{enumerate}


\newpage Table of the quantiles of the $F_{\nu _{1},\nu _{2}}$%
-distribution,	given are the 0.95 -quantiles (i.e. $a=0.05$)

\begin{tabular}{|r|rrrrrrrrrrr|}
	\hline
	& \multicolumn{11}{|c|}{$\nu_1$} \\ 
	$\nu_2$ & 1 & 2 & 3 & 4 & 5 & 10 & 15 & 20 & 25 & 50 & $\infty$\\ \hline
	1 & 161.45 & 199.50 & 215.71 & 224.58 & 230.16 & 241.88 & 245.95 & 248.01 & 249.26 & 251.77 & 254\\ 
	2 & 18.51 & 19.00 & 19.16 & 19.25 & 19.30 & 19.40 & 19.43 & 19.45 & 19.46 & 19.48 & 19.5\\ 
	3 & 10.13 & 9.55 & 9.28 & 9.12 & 9.01 & 8.79 & 8.70 & 8.66 & 8.63 & 8.58 & 8.53\\ 
	4 & 7.71 & 6.94 & 6.59 & 6.39 & 6.26 & 5.96 & 5.86 & 5.80 & 5.77 & 5.70 & 5.63\\ 
	5 & 6.61 & 5.79 & 5.41 & 5.19 & 5.05 & 4.74 & 4.62 & 4.56 & 4.52 & 4.44 & 4.37\\ 
	6 & 5.99 & 5.14 & 4.76 & 4.53 & 4.39 & 4.06 & 3.94 & 3.87 & 3.83 & 3.75 & 3.67\\ 
	7 & 5.59 & 4.74 & 4.35 & 4.12 & 3.97 & 3.64 & 3.51 & 3.44 & 3.40 & 3.32 & 3.23\\ 
	8 & 5.32 & 4.46 & 4.07 & 3.84 & 3.69 & 3.35 & 3.22 & 3.15 & 3.11 & 3.02 & 2.93\\ 
	9 & 5.12 & 4.26 & 3.86 & 3.63 & 3.48 & 3.14 & 3.01 & 2.94 & 2.89 & 2.80 & 2.71\\ 
	10 & 4.96 & 4.10 & 3.71 & 3.48 & 3.33 & 2.98 & 2.85 & 2.77 & 2.73 & 2.64 & 2.54\\ 
	15 & 4.54 & 3.68 & 3.29 & 3.06 & 2.90 & 2.54 & 2.40 & 2.33 & 2.28 & 2.18 & 2.07\\ 
	20 & 4.35 & 3.49 & 3.10 & 2.87 & 2.71 & 2.35 & 2.20 & 2.12 & 2.07 & 1.97 & 1.84\\ 
	25 & 4.24 & 3.39 & 2.99 & 2.76 & 2.60 & 2.24 & 2.09 & 2.01 & 1.96 & 1.84 & 1.71\\ 
	30 & 4.17 & 3.32 & 2.92 & 2.69 & 2.53 & 2.16 & 2.01 & 1.93 & 1.88 & 1.76 & 1.62\\ 
	35 & 4.12 & 3.27 & 2.87 & 2.64 & 2.49 & 2.11 & 1.96 & 1.88 & 1.82 & 1.70 & 1.56\\ 
	40 & 4.08 & 3.23 & 2.84 & 2.61 & 2.45 & 2.08 & 1.92 & 1.84 & 1.78 & 1.66 & 1.51\\ 
	45 & 4.06 & 3.20 & 2.81 & 2.58 & 2.42 & 2.05 & 1.89 & 1.81 & 1.75 & 1.63 & 1.47\\ 
	50 & 4.03 & 3.18 & 2.79 & 2.56 & 2.40 & 2.03 & 1.87 & 1.78 & 1.73 & 1.60 & 1.44\\ 
	60 & 4.00 & 3.15 & 2.76 & 2.53 & 2.37 & 1.99 & 1.84 & 1.75 & 1.69 & 1.56 & 1.39\\ 
	70 & 3.98 & 3.13 & 2.74 & 2.50 & 2.35 & 1.97 & 1.81 & 1.72 & 1.66 & 1.53 & 1.35\\ 
	80 & 3.96 & 3.11 & 2.72 & 2.49 & 2.33 & 1.95 & 1.79 & 1.70 & 1.64 & 1.51 & 1.32\\ 
	90 & 3.95 & 3.10 & 2.71 & 2.47 & 2.32 & 1.94 & 1.78 & 1.69 & 1.63 & 1.49 & 1.30\\ 
	100 & 3.94 & 3.09 & 2.70 & 2.46 & 2.31 & 1.93 & 1.77 & 1.68 & 1.62 & 1.48 & 1.28 \\ 
	$\infty$ & 3.84 & 3.00 & 2.60 & 2.37 & 2.21 & 1.83 & 1.67 & 1.57 & 1.49 & 1.35 & 1.01 \\ 
	\hline
\end{tabular}

	
\end{document}
